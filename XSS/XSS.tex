\documentclass[bibliography=totocnumbered]{scrartcl}

% Codificación
\usepackage[utf8]{inputenc}
% Idioma
\usepackage[spanish]{babel}
% Bibliografía
\usepackage{csquotes}
\usepackage[backend=biber,citestyle=numeric,dateabbrev=false,language=spanish,sorting=nty]{biblatex}
\addbibresource{./referencias.bib}
% Links
\usepackage{hyperref}
\hypersetup{
    colorlinks=true,
    citecolor=black,
    filecolor=black,
    linkcolor=black,
    urlcolor=black
}
% Código de programación
\usepackage{listings}
% SetUp de colores para código
\usepackage{xcolor}
\definecolor{codegreen}{rgb}{0,0.6,0}
\definecolor{codegray}{rgb}{0.5,0.5,0.5}
\definecolor{codepurple}{rgb}{0.58,0,0.82}
\definecolor{backcolour}{rgb}{0.95,0.95,0.92}
\lstdefinestyle{mystyle}{
    %backgroundcolor=\color{backcolour},   
    commentstyle=\color{codegreen},
    keywordstyle=\color{magenta},
    numberstyle=\tiny\color{codegray},
    stringstyle=\color{codepurple},
    basicstyle=\ttfamily\footnotesize,
    breakatwhitespace=false,         
    breaklines=true,                 
    captionpos=b,                    
    keepspaces=true,                 
    %numbers=left,                    
    numbersep=5pt,                  
    showspaces=false,                
    showstringspaces=false,
    showtabs=false,                  
    tabsize=2
}
\lstset{style=mystyle}
% Imágenes
\usepackage{graphicx}
% Cambios de color para links
\newcommand{\changeurlcolor}[1]{\hypersetup{urlcolor=#1}} 
\title{Cross-site Scripting (XSS)}
%\subtitle{subtitle}
\author{Víctor Nieves Sánchez}
\date{Última modificación \today{}}

\begin{document}
\maketitle
\section*{Disclaimer}
Este documento se ha elaborado por los autores, obteniendo información de diversos recursos, principalmente de la página web de \changeurlcolor{blue}\href{https://portswigger.net/web-security}{\textit{PortSwigger}}.\\
\changeurlcolor{black}
El objetivo de este documento es proporcionar una breve referencia de ayuda para el lector.\\

Si deseas más información, os recomendamos realizar los labs de su página web:
\begin{center}
\changeurlcolor{blue}\href{https://portswigger.net/web-security}{https://portswigger.net/web-security}    
\end{center}

\newpage
\tableofcontents

\newpage
\listoffigures

\newpage

\section{¿Qué es Cross-site scripting?}


\newpage
% Citar todas las ref (incluidas las que no han sido citadas)
\nocite{*}
\printbibliography 
\end{document}



