\documentclass[bibliography=totocnumbered]{scrartcl}

% Codificación
\usepackage[utf8]{inputenc}
% Idioma
\usepackage[spanish]{babel}
% Bibliografía
\usepackage{csquotes}
\usepackage[backend=biber,citestyle=numeric,dateabbrev=false,language=spanish,sorting=nty]{biblatex}
\addbibresource{./referencias.bib}
% Links
\usepackage{hyperref}
\hypersetup{
    colorlinks=true,
    citecolor=black,
    filecolor=black,
    linkcolor=black,
    urlcolor=black
}
% Código de programación
\usepackage{listings}
% SetUp de colores para código
\usepackage{xcolor}
\definecolor{codegreen}{rgb}{0,0.6,0}
\definecolor{codegray}{rgb}{0.5,0.5,0.5}
\definecolor{codepurple}{rgb}{0.58,0,0.82}
\definecolor{backcolour}{rgb}{0.95,0.95,0.92}
\lstdefinestyle{mystyle}{
    %backgroundcolor=\color{backcolour},   
    commentstyle=\color{codegreen},
    keywordstyle=\color{magenta},
    numberstyle=\tiny\color{codegray},
    stringstyle=\color{codepurple},
    basicstyle=\ttfamily\footnotesize,
    breakatwhitespace=false,         
    breaklines=true,                 
    captionpos=b,                    
    keepspaces=true,                 
    %numbers=left,                    
    numbersep=5pt,                  
    showspaces=false,                
    showstringspaces=false,
    showtabs=false,                  
    tabsize=2
}
\lstset{style=mystyle}
% Imágenes
\usepackage{graphicx}
% Cambios de color para links
\newcommand{\changeurlcolor}[1]{\hypersetup{urlcolor=#1}} 
\title{Herramientas para hacking}
%\subtitle{subtitle}
\author{Víctor Nieves Sánchez}
\date{Última modificación \today{}}

\begin{document}
\maketitle
\section*{Disclaimer}
Este documento se ha elaborado por los autores, obteniendo información de diversos recursos.\\

El objetivo de este documento es proporcionar una breve referencia de ayuda para el lector.

\newpage
\tableofcontents

\newpage
\listoffigures

\newpage
\section{Introducción}
Este documento enumera una serie de herramientas útiles para hacking.

\section{Sniffers}
Un \textbf{\textit{Sniffer}} es un programa que captura y analiza paquetes de red, tanto de entrada como de salida.
\begin{itemize}
\item \textbf{Wireshark}\parencite{wireshark}: \textit{Wireshark} es el \textit{sniffer} de red más popular, con soporte en diversas plataformas.
\item \textbf{Tcpdump}\parencite{tcpdump}: \textit{Tcpdump} es un \textit{sniffer} de linea de comandos disponible tanto en Linux como Unix.
\item \textbf{Windump}\parencite{windump}: \textit{Windump} es la versión para Windows de \textit{tcpdump}\parencite{tcpdump}.
\item \textbf{Cain \& Abel}\parencite{cain}: \textit{Cain \& Abel} es una herramienta ''todo en uno'' para capturar paquetes, grabar contraseñas usadas en un ataque \textit{MiTM}\parencite{mitm}, crackear hashes de contraseñas utilizando métodos como ataques de diccionario, de fuerza bruta y ataques basados en ''criptoanálisis'' y ataques de \textit{ARP y DNS poisoning}\parencite{arp}\parencite{dns}.
\item \textbf{Kismet}\parencite{kismet}: \textit{Kismet} es una herramienta para \textit{sniffing} de redes inalámbricas, que sirve para para localizar y descubrir SSID's ocultas. También puede ser usada como un \textit{sniffer} pasivo de tráfico.
\item \textbf{Ntop}\parencite{ntop}: \textit{Ntop} es una página web para analizar el tráfico web.
\item \textbf{Network Miner}\parencite{networkminer}: \textit{Network Miner} es un \textit{sniffer}, capaz de identificar el SO. Automáticamente extrae los archivos contenidos en los paquetes capturados, incluyendo imágenes. 
\end{itemize}

\newpage
% Citar todas las ref (incluidas las que no han sido citadas)
\nocite{*}
\printbibliography 
\end{document}

